\documentclass[12pt, letterpaper, oneside]{book}
\usepackage[margin={0.6in, 0.75in}]{geometry}
\usepackage{microtype}
% \usepackage{kpfonts}
\usepackage{amsmath, amssymb, amsthm}
\usepackage{parskip}
\usepackage[many]{tcolorbox}
\usepackage{footnote}
\usepackage{cancel}
\usepackage{titlesec}
\usepackage{pgffor}
\usepackage{mathtools}
\usepackage[shortlabels, inline]{enumitem}
\usepackage{hyperref}
\usepackage{tikz-cd}
\usepackage{bbm}

\usepackage[overload]{textcase}

\renewcommand{\chaptername}{Lecture}
\newtheorem{axiom}{Axiom}[chapter]
\newtheorem{theorem}{Theorem}[chapter]
\newtheorem{conjecture}{Conjecture}[chapter]
\newtheorem{prop}{Proposition}[chapter]
\newtheorem{corollary}{Corollary}[theorem]
\newtheorem{lemma}{Lemma}[chapter]
\theoremstyle{definition}
\newtheorem{definition}{Definition}[chapter]
\newtheorem{exercise}{Exercise}[chapter]
\newtheorem{example}{Example}[definition]
\newtheorem*{remark}{Remark}

\tcbset{sharp corners, breakable, enhanced, parbox=false}
\newtcolorbox{mybox}[3][]
{
  colframe = #2!150,
  colback  = #2!5,
  coltitle = #2!0!white,  
  title    = {#3},
  #1,
}

\titleformat{\chapter}[display]
    {\normalfont\huge\bfseries}{\chaptertitlename\ \thechapter}{20pt}{\Huge}
\titlespacing*{\chapter}{0pt}{0pt}{40pt}

\newcommand{\R}{\mathbb{R}}
\newcommand{\N}{\mathbb{N}}
\newcommand{\Z}{\mathbb{Z}}
\newcommand{\C}{\mathbb{C}}
\newcommand{\Q}{\mathbb{Q}}
\newcommand{\F}{\mathcal{F}}

\newcommand{\HH}{\mathcal{H}}

\DeclareMathOperator{\Int}{int}
\DeclareMathOperator{\re}{Re}
\DeclareMathOperator{\im}{Im}
\DeclareMathOperator{\id}{id}
\DeclareMathOperator{\Ran}{Ran}
\DeclareMathOperator{\sinc}{sinc}
\DeclareMathOperator{\supp}{supp}
\DeclareMathOperator*{\esssup}{ess\,sup}

\DeclareFontFamily{U}{mathx}{}
\DeclareFontShape{U}{mathx}{m}{n}{<-> mathx10}{}
\DeclareSymbolFont{mathx}{U}{mathx}{m}{n}
\DeclareMathAccent{\widecheck}{0}{mathx}{"71}

\newcommand{\seminorm}[1]{\left\lvert\hspace{-1 pt}\left\lvert\hspace{-1 pt}\left\lvert {#1}\right\lvert\hspace{-1 pt}\right\lvert\hspace{-1 pt}\right\lvert}

\title{MATH 7337: Harmonic Analysis}
\author{Frank Qiang\\Instructor: Christopher Heil}
\date{Georgia Institute of Technology\\Fall 2025}

\begin{document}
  \maketitle

  \begingroup
  \let\cleardoublepage\clearpage
  \tableofcontents
  \endgroup

  % \foreach \i in {00, 01, 02, 03, 04, ..., 50} {%
  %   \edef\FileName{lectures/lecture\i.tex}%     The % here are necessary to eliminate any
  %   \IfFileExists{\FileName}{%  spurious spaces that may get inserted
  %      \input{\FileName}%       at these points
  %   }
  % }
  \chapter{Aug.~19 --- The Fourier Transform}

\section{The Fourier Transform on \texorpdfstring{$L^1(\R)$}{L1(R)}}

All integrals will be taken over $\R$
unless otherwise specified.

\begin{definition}
  The \emph{Fourier transform} of
  $f \in L^1(\R)$ is
  \[
    \widehat{f}(\xi)
    = \int f(x) e^{-2\pi i \xi x}\, dx,
    \quad \xi \in \R.
  \]
\end{definition}

\begin{remark}
  Note that by the triangle inequality,
  \[
    |\widehat{f}(\xi)|
    \le \int |f(x) e^{-2\pi i \xi x}|\, dx
    = \int |f(x)|\, dx
    = \|f\|_{1}
    < \infty,
  \]
  so $\widehat{f}(\xi)$ exists
  for all $\xi \in \R$ (in fact,
  $\widehat{f}$ is continuous).
\end{remark}

\begin{remark}
  The Fourier transform is an
  operator $\F : L^1(\R) \to L^\infty(\R)$
  as
  $\|\widehat{f}\|_\infty = \esssup_{\xi \in \R} |\widehat{f}(\xi)| \le \|f\|_1$.
  This is linear in $f$.
  The \emph{operator norm} of
  $\F$ is
  \[
    \|\F\| = \|\F\|_{L^1 \to L^\infty}
    = \sup_{\|f\|_1 = 1} \|\widehat{f}\|_\infty
    \le \sup_{\|f\|_1 = 1} \|f\|_1
    = 1,
  \]
  so $\F$ is a bounded linear operator.
  However, $\mathcal{F}$ is
  not isometric (norm-preserving)
  in general.
\end{remark}

\begin{remark}
  Observe that
  \[
    \widehat{f}(0)
    = \int f(x) e^{-2\pi i \cdot 0 \cdot x}\, dx
    = \int f(x)\, dx.
  \]
  So if $f \ge 0$ and we normalize
  $f$ so that $\widehat{f}(0) = 1$, then
  we have
  \[
    |\widehat{f}(\xi)|
    \le \int f(x)\, dx = \widehat{f}(0),
  \]
  and so $\|\widehat{f}\|_\infty = \esssup_{\xi \in \R} |\widehat{f}(\xi)| \le 1$.
  This is one particular
  case where $\F$ does preserve the norm.
\end{remark}

\begin{definition}
  For $r \ne 0$, \emph{dilation} of
  $f$ by $r$ is $f_r(x) = rf(rx)$.
  Note that $\|f_r\|_1 = \|f\|_1$.
\end{definition}

\begin{example}
  The \emph{Dirichlet function}
  is $d(\xi) = \sin(\xi) / \pi \xi \in C_0(\R)$.\footnote{Recall that $C_0(\R)$ is the space of continuous functions $f : \R \to \C$ such that $\lim_{x \to \pm \infty} f(x) = 0$.}
  Note that $d \notin L^1(\R)$.
  We can also define the
  \emph{sinc} function as
  $\sinc \xi = \sin(\pi \xi) / (\pi \xi) = d\pi(x)$.

  However, $d$ is the Fourier transform
  of a function in $L^1(\R)$.
  Consider the
  \[
    \chi_{-[T, T]}(x) =
    \begin{cases}
      1 & \text{if } |x| \le T, \\
      0 & |x| > T.
    \end{cases}
  \]
  Note that $\chi_{-[T, T]} \in L^1(\R)$.
  Then we have
  \[
    \widehat{\chi}_{-[T, T]}(\xi)
    = \int_{-T}^{T} e^{-2\pi i \xi x}\, dx
    = \left.\frac{e^{-2\pi i \xi x}}{-2\pi i \xi}\right|_{-T}^T
    = \frac{\sin(2\pi T \xi)}{\pi \xi}
    = d_{2\pi T}(\xi),
  \]
  so we see that
  $\widehat{\chi}_{-[T, T]} \in C_0(\R) \subsetneq L^\infty(\R)$.
\end{example}

\begin{remark}
  We will see in general that
  $\mathcal{F} : L^1(\R) \to C_0(\R)$,
  this is the Riemann-Lebesgue lemma.
  The image of $\F$ is a proper
  dense subspace of $C_0(\R)$, which
  implies that $\F^{-1}$ must be unbounded
  as a linear operator by Banach space
  theory.
\end{remark}

\begin{prop}
  If $f \in L^1(\R)$, then $\widehat{f}$
  is uniformly continuous on $\R$, i.e.
  \[
    \|\widehat{f} - T_\eta\widehat{f}\|_\infty
    = \sup_{\xi \in \R} |\widehat{f}(\xi) - \widehat{f}(\xi - \eta)|
    \xrightarrow{\eta \to 0} 0,
  \]
  where $T_\eta \widehat{f}(\xi) = \widehat{f}(\xi - \eta)$.
\end{prop}

\begin{proof}
  We can write the difference as
  \[
    |\widehat{f}(\xi) - \widehat{f}(\xi - \eta)|
    = \left|\int f(x) (e^{-2\pi i \xi x} - e^{-2\pi i (\xi - \eta)x})\, dx\right|
    \le \int |f(x)| |e^{-2\pi i \xi x} - e^{-2\pi i (\xi - \eta)x}|\, dx.
  \]
  Note that $f \in L^1(\R)$ and
  $|e^{-2\pi i \xi x} - e^{-2\pi i (\xi - \eta)x}| = |1 - e^{2\pi i \eta x}| \to 0$
  as $\eta \to 0$
  independent of $\xi$, so the statement
  follows from the dominated convergence
  theorem (the integrand is dominated
  by $2f$).
\end{proof}

\section{Motivation for the Fourier Transform}

\begin{remark}
  We will define the \emph{inverse Fourier transform}
  of $f \in L^1(\R)$ as
  \[
    \widecheck{f}(x)
    = \int f(x) e^{2\pi i \xi x}\, d\xi.
  \]
  Note that $\widecheck{f}(\xi) = \widehat{f}(-\xi)$.
  With enough assumptions, this is
  an inverse to the Fourier transform.
\end{remark}

\begin{prop}[Fourier inversion formula]
  If $f, \widehat{f} \in L^1(\R)$,
  then
  \[
    f(x) = (\widehat{f})^\vee(x)
    = \int \widehat{f}(\xi) e^{2\pi i \xi x}\, d\xi.
  \]
\end{prop}

\begin{remark}
  Note that $e_\xi(x) = e^{2\pi i \xi x} = \cos 2\pi \xi x + i \sin 2\pi \xi x$
  and $e_\xi : \R \to S^1 = \{ z \in \C : |z| = 1 \}$.
  We have
  $e_\xi(x + y) = e_\xi(x) e_\xi(y)$, so
  $e_\xi$ is a homomorphism, and it is
  also continuous. Thus $e_\xi$ is
  a \emph{character} on $\R$ (in fact,
  every character on $\R$ is of the form
  $e_\xi$ for some $\xi$). One
  can use this idea to define Fourier
  transforms in much more general settings.
\end{remark}

\begin{remark}
  The Fourier transform decomposes a
  function $f$ into the pure harmonics
  $e_\xi$, and the inversion formula says
  that we can recover $f$ as a ``sum''
  of these pure harmonics.
\end{remark}

  \chapter{Aug.~21 --- The Riemann-Lebesgue Lemma}

\section{Properties of the Fourier Transform}

\begin{definition}
  Define the following operators:
  \begin{enumerate}
    \item \emph{Translation}:
      $T_a f(x) = f(x - a)$ for $a \in \R$;
    \item \emph{Modulation}:
      $M_b f(x) = e^{2\pi i b x} f(x)$
      for $b \in \R$;
    \item \emph{Dilation}:
      $f_{\lambda}(x) = \lambda f(\lambda x)$
      for $\lambda > 0$;
    \item \emph{Involution}:
      $\widetilde{f}(x) = \overline{f(-x)}$.
  \end{enumerate}
\end{definition}

\begin{remark}
  Translation and modulation are
  isometries on $L^p(\R)$ for any $p$.
  Dilation
  as defined above is $L^1$-normalized, so
  it is only an isometry on $L^1(\R)$.
\end{remark}

\begin{exercise}
  If $f \in L^1(\R)$, then
  \begin{enumerate}
    \item $(T_a f)^\wedge(\xi) = (M_{-a} \widehat{f})(\xi) = e^{-2\pi i \xi a} \widehat{f}(\xi)$;
    \item $(M_b f)^\wedge(\xi) = (T_b \widehat{f})(\xi) = \widehat{f}(\xi - b)$;
    \item $(f_\lambda)^\wedge(\xi) = \lambda (f_{1 / \lambda})^\wedge(\xi) = \widehat{f}(\xi / \lambda)$;\footnote{Note that the result is an $L^\infty$-normalized dilation.}
    \item $(\overline{f})^\wedge(\xi) = (\widehat{f})^{\sim}(\xi) = \overline{\widehat{f}(-\xi)}$;
    \item $(\widetilde{f})^\wedge(\xi) = \overline{\widehat{f}(\xi)}$.
  \end{enumerate}
\end{exercise}

\section{The Riemann-Lebesgue Lemma}
\begin{definition}
  Let $C_c(\R)$ be the space of continuous
  functions with compact support. For a continuous function, the \emph{support} of $f$, denoted $\supp(f) = \overline{\{x \in \R : f(x) \ne 0\}}$.
  So for a continuous function $f$,
  $\supp(f)$ is compact if and only if
  $f = 0$ outside some finite interval.
\end{definition}

\begin{theorem}
  $C_c(\R)$ is dense in $L^p(\R)$
  for $1 \le p < \infty$. In other words,
  \begin{enumerate}
    \item the closure of $C_c(\R)$ in $L^p(\R)$
      is all of $L^p(\R)$;
    \item for any $f \in L^p(\R)$
      and $\epsilon > 0$, there
      exists $g \in C_c(\R)$ such that
      $\|f - g\|_p < \epsilon$;
    \item if $f \in L^p(\R)$, then there
      exists $g_n \in C_c(\R)$ such that
      $g_n \to f$ in $L^p$-norm, i.e.
      $\|g_n - f\|_p \to 0$.
  \end{enumerate}
  For $p = \infty$, $C_c(\R)$ is
  dense in $C_0(\R)$ with respect to
  the $L^\infty$-norm (this is the same
  as the uniform norm for continuous
  functions).
\end{theorem}

\begin{proof}
  We sketch the proof. First approximate
  $f \in L^p(\R)$ by a simple
  function (one that takes only finitely
  many distinct values)
  $\phi = \sum_{k = 1}^N c_k \chi_{E_k}$,
  e.g. by rounding down to the nearest
  integer multiple of $2^{-n}$. Then
  use Urysohn's lemma to approximate
  $\chi_{E_k}$ by a continuous function.
\end{proof}

\begin{exercise}
  Fix $1 \le p < \infty$. Prove that
  if $f \in L^p(\R)$, then
  $\lim_{a \to 0} \|f - T_a f\|_p = 0$.
  We say that translation is
  \emph{strongly continuous} on $L^p(\R)$.
  For $p = \infty$, use
  $C_0(\R)$ and the uniform norm instead.
\end{exercise}

\begin{lemma}[Riemann-Lebesgue lemma]
  If $f \in L^1(\R)$, then
  $\widehat{f} \in C_0(\R)$,
\end{lemma}

\begin{proof}
  We have already seen that $\widehat{f}$
  is continuous. So it
  suffices to show decay at $\infty$.
  Write
  \[
    \widehat{f}(\xi)
    = - \int f(x) e^{-2\pi i \xi x}
    e^{-2\pi i \xi (1 / 2\xi)}\, dx
    = -\int_{-\infty}^\infty f(x) e^{-2\pi i \xi (x + 1 / 2\xi)}\, dx.
  \]
  Now make the change of variables
  $x \mapsto x - 1 / 2\xi$, so we get
  \[
    \widehat{f}(\xi)
    = - \int_{-\infty}^\infty f\left(x - \frac{1}{2\xi}\right) e^{-2\pi i \xi x}\, dx
    = - \int T_{1 / 2\xi} f(x) e^{-2\pi i \xi x}\, dx.
  \]
  Taking an average with the usual
  expression for $\widehat{f}(\xi)$, we
  have
  \[
    \widehat{f}(\xi)
    = \frac{1}{2} \int (f(x) - T_{1 / 2\xi} f(x)) e^{-2\pi i \xi x}\, dx.
  \]
  Taking absolute values, we obtain
  \[
    |\widehat{f}(\xi)|
    \le \frac{1}{2} \int |f(x) - T_{1 / 2\xi} f(x)|\, dx
    = \frac{1}{2} \|f - T_{1 / 2\xi} f\|_1
    \xrightarrow[\xi \to \pm \infty]{}
    0
  \]
  by the strong continuity of translation
  on $L^1(\R)$.
\end{proof}

\begin{exercise}
  The following is an alternative proof
  of the Riemann-Lebesgue lemma.
  Recall that we have $\widehat{\chi}_{-T, T} = d_{2\pi T} \in C_0(\R)$.
  By taking translations and dilations, we
  see that $\widehat{\chi}_{[a, b]} \in C_0(\R)$.
  Consider \emph{really simple functions}
  $\phi = \sum_{k = 1}^N c_k \chi_{[a_k, b_k]}$,
  and by linearity we can write
  \[
    \widehat{\phi}
    = \sum_{k = 1}^N c_k \widehat{\chi}_{[a_k, b_k]} \in C_0(\R).
  \]
  Note that really simple functions are also
  dense in $L^p(\R)$ for $1 \le p < \infty$.
  So if $f \in L^1(\R)$, there exist
  really simple $\phi_n \to f$ in $L^1$-norm.
  On the Fourier side, we have
  \[
    \|\widehat{f} - \widehat{\phi}_n\|_\infty
    \le \|f - \phi_n\|_1 \longrightarrow 0.
  \]
  Since $\phi_n \to \widehat{f}$ uniformly
  and $C_0(\R)$ is a Banach space,
  we conclude $\widehat{f} \in C_0(\R)$.
  Fill in the details.
\end{exercise}

\section{Position and Momentum Operators}

\begin{definition}
  The \emph{position operator} $P : L^1(\R) \to L^1(\R)$ is
  given by $P f(x) = x f(x)$.
  Note that $P$ is unbounded on $L^1(\R)$
  (in fact, $P$ is not defined on all
  of $L^1(\R)$).
  Restrict $P$ to the domain
  \[
    D_P = \{f \in L^1(\R) : xf(x) \in L^1(\R)\},
  \]
  which is dense in $L^1(\R)$.
  Note that $D_P$ cannot be bounded
  as it does not admit an extension
  to $L^1(\R)$.
\end{definition}

\begin{exercise}
  Show that
  $\sup_{\|f\|_1 = 1, f \in D_P} \|P f\|_1 = \infty$.
\end{exercise}

\begin{definition}
  The \emph{momentum operator}
  $M : L^1(\R) \to L^1(\R)$ is given by
  $M f = f' / 2\pi i$. Similarly,
  $M$ is unbounded and defined only on a
  dense subset of $L^1(\R)$.
\end{definition}

\begin{remark}
  We have the relation
  $(Mf)^\wedge(\xi) = \xi P \widehat{f}(\xi)$,
  whenever the statement makes sense.
\end{remark}

\section{The HRT Conjecture}

\begin{conjecture}[HRT conjecture]
  Assume $g$ is not zero a.e.,
  $a_k, b_k$ are distinct, and
  consider finite linear combinations of
  translations and modulations
  of $g \in L^2(\R)$ of the following form:
  \[
    \sum_{k = 1}^N c_k e^{2\pi i b_k x} g(x - a_k). \tag{$*$}
  \]
  If $(*) = 0$, then must it be that
  $c_1 = \cdots = c_N = 0$? In other
  words, are these linearly
  independent?
\end{conjecture}

\begin{remark}
  Consider the special case $b_k = 0$
  for every $k$, so
  $\sum c_k T_{a_k} g = \sum c_k g(x - a_k) = 0$ a.e. Then
  \[
    \left(\sum c_k T_{a_k} g\right)^\wedge
    = \sum c_k M_{-a_k} \widehat{g}
    = \left(\sum_{k = 1}^N c_k e^{-2\pi i a_k \xi}\right)
    \widehat{g}(\xi) = 0.
  \]
  Since $\widehat{g}$ is not zero a.e., we
  must have
  $\sum_{k = 1}^N c_k e^{-2\pi i a_k \xi} = 0$,
  which implies $c_k = 0$ for all $k$.
  In particular, this means that
  translations alone are linearly independent
  (the same is true for modulations alone).
\end{remark}

\begin{remark}
  The general case of the HRT conjecture
  is still open. Note that after taking a
  Fourier transform, we end up with
  the same problem, just for $\widehat{g}$
  instead of $g$.
\end{remark}

\end{document}
