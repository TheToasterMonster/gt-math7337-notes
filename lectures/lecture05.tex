\chapter{Sept.~2 --- Smoothness and Decay}

\section{Smoothness and Decay}

\begin{theorem}[Decay in time implies smoothness in frequency]
  Assume $f \in L^1(\R)$ and
  $x^m f(x) \in L^1(\R)$, where $m > 0$.
  Then
  \[
    \widehat{f}
    \in C_0^m(\R)
    = \{
      g : g, g', \dots, g^{(m)} \in C_0(\R)
    \}.
  \]
  Furthermore, we have
  \[
    \widehat{f}^{(k)}
    = \frac{d^k}{d\xi^k} \widehat{f}
    = \left((-2\pi i x)^k f(x)\right)^\wedge.
  \]
\end{theorem}

\begin{proof}
  The proof is by induction on $m$.
  When $m = 1$, we can formally write
  \begin{align*}
    \frac{d}{d\xi} \widehat{f}(\xi)
    &= \frac{d}{d\xi} \int f(x) e^{-2\pi i \xi x} \, dx \\
    &\overset{(*)}{=} \int f(x) \frac{d}{d\xi} e^{-2\pi i \xi x} \, dx
    = \int f(x) (-2\pi i x) e^{-2\pi i \xi x} \, dx
    = (-2\pi i x f(x))^\wedge(\xi).
  \end{align*}
  It suffices to justify step $(*)$, which
  we will do by appealing to the
  dominated convergence theorem. We
  can write
  \[
    \widehat{f}'(\xi)
    = \lim_{\eta \to 0} \frac{\widehat{f}(\xi + \eta) - \widehat{f}(\xi)}{\eta}
    = \lim_{\eta \to 0} \int f(x) \frac{e^{-2\pi i (\xi + \eta) x} - e^{-2\pi i \xi x}}{\eta}\, dx.
  \]
  Note that we have the pointwise limit
  \[
    f(x) \frac{e^{-2\pi i (\xi + \eta) x} - e^{-2\pi i \xi x}}{\eta}
    \xrightarrow{\eta \to 0}
    f(x) \frac{d}{d\xi} e^{-2\pi i \xi x}
    = -2\pi i x f(x) e^{-2\pi i \xi x}.
  \]
  Also note that we can bound
  \[
    \left|f(x) \frac{e^{-2\pi i (\xi + \eta) x} - e^{-2\pi i \xi x}}{\eta}\right|
    = \left|f(x) \frac{e^{-2\pi i \eta x} - 1}{\eta}\right|
    \le \left|f(x) \frac{-2\pi i \eta x}{\eta}\right|
    = |2\pi x f(x)|,
  \]
  where we noted that $|e^{i\theta} - 1| \le |\theta|$ for $\theta \in \R$.
  Thus $2\pi x f(x)$ dominates the
  integrand and is integrable since
  $x f(x) \in L^1(\R)$ by assumption,
  we can conclude $(*)$ by the dominated
  convergence theorem. Then
  $\widehat{f}' \in C_0(\R)$ by the
  Riemann-Lebesgue lemma, since
  $\widehat{f}' = (-2\pi i x f(x))^\wedge$
  where $-2\pi i x f(x) \in L^1(\R)$.

  The inductive step is part of Homework 1.
\end{proof}

\begin{remark}
  Recall the  position and momentum
  operators $Pf(x) = x f(x)$ and
  $M f(x) = f'(x) / 2\pi i$. If
  $f, Pf \in L^1(\R)$, then the above
  theorem tells us that
  $(Pf)^\wedge = -M \widehat{f}$.
\end{remark}

\section{Absolute Continuity}

\begin{definition}
  A function $f : [a, b] \to \C$ is
  \emph{absolutely continuous} if
  for all $\epsilon > 0$, there exists
  $\delta > 0$ such that if
  $\{[a_j, b_j]\}_{j}$ are countably many
  non-overlapping intervals, then
  \[
    \sum_j (b_j - a_j) < \delta
    \quad \text{implies} \quad
    \sum_j |f(b_j) - f(a_j)| < \epsilon.
  \]
  Define $\AC_{\loc}(\R) = \{f \in C(\R) : f \text{ is absolutely continuous on every interval } [a, b]\}$.
\end{definition}

\begin{theorem}[Fundamental theorem of calculus]
  If $g : [a, b] \to \C$, then
  the following are equivalent:
  \begin{enumerate}
    \item $g \in \AC[a, b]$;
    \item there exists $f \in L^1[a, b]$
      such that for all $x \in [a, b]$,
      \[
        g(x) - g(a)
        = \int_a^x f(t) \, dt;
      \]
    \item $g$ is differentiable at a.e.
      point, $g' \in L^1[a, b]$, and
      \[
        g(x) - g(a)
        = \int_a^x g'(t) \, dt.
      \]
  \end{enumerate}
\end{theorem}

\begin{remark}
  The Cantor-Lebesgue function
  $\varphi : [0, 1] \to [0, 1]$ is continuous
  with $\varphi' = 0$ a.e., but
  \[
    \int_0^1 \varphi'(x) \, dx = 0
    \ne 1 = \varphi(1) - \varphi(0).
  \]
\end{remark}

\begin{lemma}[Growth lemma]
  If $f : [a, b] \to \R$ is measurable and
  differentiable at every point in
  a measurable set $E \subseteq [a, b]$,
  then
  \[
    |f(E)|_e \le \int_E |f'|,
  \]
  where $|f(E)|_e$ denotes the
  exterior Lebesgue measure of $f(E)$.
\end{lemma}

\begin{theorem}[Banach-Zaretsky theorem]
  If $f : [a, b] \to \R$, then the
  following are equivalent:
  \begin{enumerate}
    \item $f \in \AC[a, b]$;
    \item $f$ is continuous, $f$ has
      bounded variation, and
      $|A| = 0$ implies $|f(A)| = 0$;
    \item $f$ is continuous and
      differentiable a.e.,
      $f' \in L^1[a, b]$, and
      $|A| = 0$ implies $|f(A)| = 0$.
  \end{enumerate}
\end{theorem}

\begin{theorem}\label{thm:abs-cont-sufficient}
  If $f : [a, b] \to \C$ is differentiable
  on $[a, b]$ and $f' \in L^1[a, b]$,
  then $f \in \AC[a, b]$.
\end{theorem}

\begin{proof}
  By the Banach-Zaretsky theorem, it
  suffices to show that $|A| = 0$ implies
  $|f(A)| = 0$. If $|A| = 0$, then by the
  growth lemma,
  \[
    |f(A)| \le \int_A |f'| = 0,
  \]
  which completes the proof. (Technically
  we should split $f$ into its real and
  imaginary parts.)
\end{proof}

\section{Smoothness and Decay, Continued}

\begin{theorem}[Smoothness in time implies decay in frequency]
  If $f \in L^1(\R)$ is everywhere
  $m$-times differentiable
  and $f, f', \dots, f^{(m)} \in L^1(\R)$,
  then
  \[
    \widehat{f^{(k)}}(\xi)
    = (2\pi i \xi)^k \widehat{f}(\xi),
    \quad \text{for } k = 0, \dots, m,
  \]
  hence $|\widehat{f}(\xi)| \le |2\pi \xi|^{-k} |\widehat{f^{(k)}}(\xi)| \le |2\pi \xi|^{-k} \|\widehat{f^{(k)}}\|_\infty \le |2\pi \xi|^{-k} \|f^{(k)}\|_1$
  for $k = 0, \dots, m$.
\end{theorem}

\begin{proof}
  We prove only the case $m = 1$, the
  rest follows by induction. Assume
  $f, f' \in L^1(\R)$. By Theorem
  \ref{thm:abs-cont-sufficient},
  we have $f \in \AC_{\loc}(\R)$. Hence
  by the fundamental theorem of calculus,
  \[
    f(x) - f(0) = \int_0^x f'(t) \, dt.
  \]
  Because $f'$ is integrable, we get that
  \[
    \lim_{x \to \infty} f(x)
    =
    f(0) + \lim_{x \to \infty} \int_0^x f'(t)\, dt
    = f(0)  + \int_0^\infty f'(t) \, dt.
  \]
  Since $f$ is integrable and this limit
  exists, the limit must be $0$. Hence
  $f \in C_0(\R)$. We can compute
  \[
    \widehat{f'}(\xi)
    = \int_{-\infty}^\infty f'(x) e^{-2\pi i \xi x} \, dx
    = \lim_{\substack{b \to \infty \\ a \to -\infty}} \int_a^b f'(x) e^{-2\pi i \xi x} \, dx.
  \]
  Since $f$ is absolutely continuous, we
  can integrate by parts to get
  \[
    \widehat{f'}(\xi)
    = \lim_{\substack{b \to \infty \\ a \to -\infty}}
    \left[f(b) e^{-2\pi \xi b} - f(a) e^{-2\pi i \xi a} + (2\pi i \xi)\int_a^b f(x) e^{-2\pi i \xi x}\, dx \right]
    = (2\pi i \xi) \widehat{f}(\xi),
  \]
  which proves the desired result.
\end{proof}

\begin{remark}
  Note that for the absolute continuity
  arguments, we need to first restrict
  to a finite interval and then
  take limits, since we only know that
  $f \in \AC_{\loc}(\R)$.
\end{remark}

\section{Approximate Identities}

\begin{remark}
  Recall that if we take
  $g_T = \chi_{[-T, T]} / 2T$,
  then we have
  $(f * g_T)(x) = \mathrm{Avg}_{[x - T, x + T]} f$.
  As $T \to 0$, this converges to
  $f$ if $f$ is continuous, and converges
  a.e. to $f$ if $f$ is integrable. In
  particular, this is almost like
  a identity for the convolution operation.
\end{remark}

\begin{definition}
  If $k_\lambda \in L^1(\R)$ for $\lambda > 0$ (or sometimes $\lambda \in \N$) satisfy:
  \begin{enumerate}[(a)]
    \item Normalization: $\displaystyle \int_{-\infty}^\infty k_{\lambda} = 1$ for every $\lambda$,
    \item $L^1$-boundedness:
      $\displaystyle \sup_{\lambda} \|k_\lambda\|_1 = \sup_{\lambda} \int_{-\infty}^\infty |k_\lambda| < \infty$,
    \item $L^1$-concentration:
      $\displaystyle \lim_{\lambda \to \infty} \int_{|x| \ge \delta} |k_\lambda| = 0$
      for every $\delta > 0$,
  \end{enumerate}
  then we say that $\{k_\lambda\}$ is
  an \emph{approximate identity (for
  convolution)}.
\end{definition}

\begin{exercise}
  If $k \in L^1(\R)$ and
  $\displaystyle \int_{-\infty}^\infty k = 1$, then
  $k_\lambda(x) = \lambda k(\lambda x)$
  forms an approximate identity.
\end{exercise}

\begin{remark}
  If we choose $k_{\lambda}$ to be nice,
  then $f * k_\lambda$ will also be nice and
  ``close'' to $f$.
\end{remark}
