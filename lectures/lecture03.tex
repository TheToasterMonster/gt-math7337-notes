\chapter{Aug.~3 --- Convolution}

\section{Convolution}

\begin{definition}
  If $f, g$ are measurable on $\R$, their
  \emph{convolution} is (formally)
  \[
    (f * g)(x)
    = \int_{-\infty}^\infty
    f(y) g(x - y)\, dy.
  \]
\end{definition}

\begin{remark}
  When it exists, we have
  \[
    (f * g)(x)
    = \int_{-\infty}^\infty
    f(y) g(x - y)\, dy
    = \int_{-\infty}^\infty
    f(x - y) g(y)\, dy
    = (g * f)(x)
  \]
  by the change of variables
  $y \mapsto x - y$. So
  $f * g = g * f$, if it exists. Similarly,
  $f * (g * h) = (f * g) * h$
  if each of these convolutions exist.
\end{remark}

\begin{remark}
  If we take $g_T = \chi_{-T, T} / 2T$
  (note that $\|g_T\|_1 = 1$),
  then
  \[
    (f * g_T)(x)
    = \int_{-\infty}^\infty
    f(y) g_T(x - y)\, dy
    = \frac{1}{2T} \int_{x - T}^{x + T} f(y) dy
    = \mathrm{Avg}_{[-T, T]} f(x),
  \]
  so we can see convolution as a
  averaging or smoothing operation
  (also known as \emph{mollification}).
\end{remark}

\begin{remark}
  We would like to show
  $f, g \in L^1(\R)$ implies
  $f * g \in L^1(\R)$. Note that
  $(f * g)^\wedge = \widehat{f} \widehat{g} \in C_0(\R)$,
  since $C_0(\R)$ is closed under
  multiplication, even though $L^1(\R)$
  is not.
\end{remark}

\begin{remark}
  The \emph{Lebesgue differentiation theorem}
  says that if $f \in L^1_{\loc}(\R)$, then
  $(f * g_T)(x) \to f(x)$ a.e.
\end{remark}

\section{Properties of Convolution}

\begin{remark}
  Use the notation
  \[
    \langle f, g \rangle
    = \int_{-\infty}^\infty
    f(x) \overline{g(x)}\, dx,
  \]
  whenever this integral exists.
  Then \emph{H\"older's inequality}
  says that if $1 / p + 1 / p' = 1$
  with $1 \le p \le \infty$ and
  $f \in L^p(\R)$, $g \in L^{p'}(\R)$,
  then $fg \in L^1(\R)$ and we have
  \[
    |\langle f, g \rangle|
    \le \int |f(x)| |g(x)|\, dx
    \le \|f\|_p \|g\|_{p'}.
  \]
\end{remark}

\begin{theorem}
  For $1 \le p \le \infty$, if
  $f \in L^p(\R)$ and
  $g \in L^1(\R)$, then
  $f * g \in L^\infty(\R)$.
\end{theorem}

\begin{proof}
  By H\"older's inequality, we can write
  \[
    \int |f(y) g(x - y)|\, dy
    \le \|f\|_p \|g(x)\|_{p'}
    < \infty,
  \]
  so $(f * g)(x)$ exists for every $x \in \R$.
\end{proof}

\begin{exercise}
  Show that $f * g \in C_b(\R) = \{h : \R \to \C : h \text{ is continuous and bounded}\}$.
\end{exercise}

\begin{remark}
  Denote $g^*(y) = \overline{g(-y)}$. Then
  we have
  \[
    (f * g)(x)
    = \int f(y) g(x - y)\, dy
    = \int f(y) \overline{g^*(y - x)}\, dy
    = \langle f, T_x g^* \rangle.
  \]
\end{remark}

\begin{theorem}
  Let $f, g \in L^1(\R)$. Then
  \begin{enumerate}
    \item $f(y) g(x - y)$ is measurable
      and integrable on $\R^2$;
    \item for a.e. $x \in \R$,
      $f(y) g(x - y)$ is measurable
      and integrable on $\R$ as a
      function of $y$;
    \item $f * g \in L^1(\R)$ and
      $\|f * g\|_1 \le \|f\|_1 \|g\|_1$, i.e.
      convolution is
      \emph{submultiplicative} on $L^1(\R)$;
    \item $(f * g)^\wedge(\xi) = \widehat{f}(\xi) \widehat{g}(\xi)$ for every $\xi \in \R$.
  \end{enumerate}
\end{theorem}

\begin{proof}
  $(1)$ Let $h(x, y) = f(x)$. Then we have
  \[
    \{h > a\} = h^{-1}((a, \infty))
    = \{(x, y) : f(x) > a\}
    = \{f > a\} \times \R,
  \]
  which is measurable in $\R^2$ since
  $\{f > a\}$ and $\R$ are measurable in
  $\R$. Similarly,
  $g(y)$ is measurable on $\R^2$, so
  $F(x, y) = f(x) g(y)$ is measurable
  on $\R^2$. Now make a linear change
  of variables $T(x, y) = (y, x - y)$,
  so $H = F \circ T = f(y) g(x - y)$ is
  measurable (note that linear maps
  preserve measurability).

  Now we can integrate by Tonelli's
  theorem and see that
  \begin{align*}
    \iint |f(y) g(x - y)|\, dx dy
    &= \int |f(y)| \left(\int |g(x - y)|\, dx\right) dy
    = \int |f(y)| \left(\int |g(z)|\, dz\right) dy \\
    &= \int |f(y)| \|g\|_1\, dy
    = \|f\|_1 \|g\|_1 < \infty,
  \end{align*}
  hence $f(y) g(x - y)$ is integrable
  on $\R^2$.

  $(2)$ This follows by Fubini's theorem
  since $f(y) g(x - y)$ is integrable.

  $(3)$ By $(2)$, $(f * g)(x)$ exists
  for a.e. $x$, and
  \[
    \int |(f * g)(x)|\, dx
    = \int \left|\int f(y) g(x - y)\, dy\right| dx
    \le \iint |f(y) g(x - y)|\, dy dx
    \le \|f\|_1 \|g\|_1,
  \]
  which is the desired inequality.

  $(4)$ Justify the following calculation
  as an exercise via Fubini/Tonelli's theorem:
  \begin{align*}
    (f * g)^\wedge(\xi)
    &= \int (f * g)(x) e^{-2\pi i \xi x}\, dx
    = \int \left(\int f(y) g(x - y)\, dy\right) e^{-2\pi i \xi x}\, dx \\
    &= \iint f(y) e^{-2\pi i \xi y}
    g(x - y) e^{-2\pi i \xi (x - y)}\, dy dx.
  \end{align*}
  By Fubini's theorem, we can exchange
  orders and write
  \begin{align*}
    (f * g)^\wedge(\xi)
    &= \int f(y) e^{-2\pi i \xi y}
    \left(\int g(x - y) e^{-2\pi i \xi (x - y)}\, dx\right) dy \\
    &= \int f(y) e^{-2\pi i \xi y} \left(\int g(z) e^{-2\pi i \xi z}\, dz\right) dy
    = \widehat{f}(\xi) \widehat{g}(\xi),
  \end{align*}
  which is the desired equality.
\end{proof}

\begin{corollary}
  $L^1(\R)$ is closed under convolution.
\end{corollary}

\begin{definition}
  An \emph{algebra} is a vector space
  $A$ with a product such that
  \begin{enumerate}[(a)]
    \item $(fg) h = f(gh)$,
    \item $f(g + h) = fg + fh$,
    \item $\alpha (fg) = (\alpha f) g = f(\alpha g)$.
  \end{enumerate}
  If $fg = gf$ always, then we say that
  $A$ is \emph{commutative}.
  A Banach space which is also an algebra
  with a submultiplicative product
  is a \emph{Banach algebra}.
\end{definition}

\begin{example}
  With convolution as a product,
  $L^1(\R)$ becomes a commutative
  Banach algebra without identity.
  Similarly, $C_0(\R)$ is also a
  commutative Banach algebra
  without identity (under pointwise products).
  The space $\mathcal{B}(X)$ of
  bounded linear operators on a Banach
  space $X$ is also a Banach space under the
  operator norm, and
  we have $\|AB\| \le \|A\| \|B\|$
  with composition as a product.
  So $\mathcal{B}(X)$ is a noncommutative
  Banach algebra, with identity.
\end{example}

\section{Young's Inequality}
\begin{theorem}[Young's inequality, special case]
  Fix $1 \le p \le \infty$. If
  $f \in L^p(\R)$ and $g \in L^1(\R)$, 
  then $f * g \in L^p(\R)$ and
  $\|f * g\|_p \le \|f\|_p \|g\|_1$.
\end{theorem}

\begin{proof}
  The case $p = \infty$ is easy
  by H\"older's inequality and
  $p = 1$ is done,
  so assume $1 < p < \infty$. Then
  \[
    |(f * g)(x)|
    \le \int |f(y)| |g(x - y)|\, dy
    = \int \left(|f(y)| |g(x - y)|^{1 / p}\right) \left(|g(x - y)|^{1 / p'}\right) \, dy,
  \]
  By H\"older's inequality, we can write
  \begin{align*}
    |(f * g)(x)|
    &\le \left(\int |f(y)|^p |g(x - y)|\, dy\right)^{1 / p}
    \left(\int |g(x - y)|\, dy\right)^{1 / p'} \\
    &\le \|g\|_1^{1 / p'} \left(\int |f(y)|^p |g(x - y)|\, dy\right)^{1 / p}.
  \end{align*}
  Now taking $L^p$-norms, we get
  \[
    \|f * g\|_p^p
    = \int |(f * g)(x)|^p\, dx
    \le \|g\|_1^{p / p'}
    \iint |f(y)|^p |g(x - y)|\, dy dx.
  \]
  By Tonelli's theorem, we can exchange
  orders and write
  \[
    \|f * g\|_p^p
    \le \|g\|_1^{p / p'}
    \int |f(y)|^p \left(\int |g(x - y)|\, dx\right) dy
    \le \|g\|_1^{1 + p / p'} \|f\|_p^p
    = \|g\|_1^p \|f\|_p^p,
  \]
  so we get the desired inequality
  $\|f * g\|_p \le \|f\|_p \|g\|_1$
  after taking $p$th roots.
\end{proof}

\begin{exercise}[Young's inequality, general case]
  Let $1 \le p, q, r \le \infty$ satisfy
  $1 / r = 1 / p + 1 / q - 1$.
  If $f \in L^p(\R)$ and
  $g \in L^q(\R)$, then
  \[
    \|f * g\|_r \le \|f\|_p \|g\|_q.
  \]
\end{exercise}

\begin{remark}
  Recall \emph{Minkowski's inequality}
  (the triangle inequality in $L^p(\R)$):
  \[
    \left\| \sum f_k \right\|_p
    \le \sum \|f_k\|_p.
  \]
  \emph{Minkowski's integral inequality}
  then says that for $1 \le p \le \infty$,
  \[
    \left\|\int f_x\, dx \right\|_p
    = \left(\int \left|\int f(x, y)\, dx\right|^p\, dy\right)^{1 / p}
    \le \int \left(\int |f(x, y)|^p\, dy\right)^{1 / p} dx
    = \int \|f_x\|_p\, dx.
  \]
  One can also use this to prove to
  Young's inequality.
\end{remark}

\begin{remark}
  The \emph{Babenko-Beckner constant}
  is the optimal constant in front
  of H\"older's inequality:
  \[
    A_p = \left(\frac{p^{1 / p}}{(p')^{1 / p'}}\right)^{1 / 2}.
  \]
  The optimal constant in Young's inequality
  is $A_p A_q A_{r'}$, i.e. we have
  \[
    \|f * g\|_r
    \le (A_p A_q A_{r'}) \|f\|_p \|g\|_q.
  \]
\end{remark}

\section{The Dirac Delta}

\begin{remark}
  Is there an identity for convolution?
  Suppose there was a function
  $\delta \in L^1(\R)$ (the \emph{Dirac delta function}) such that
  $f * \delta = f$ for all
  $f \in L^1(\R)$. Then we have
  $(f * \delta)^\wedge = \widehat{f}$,
  so
  \[
    \widehat{f}(\xi) \widehat{\delta}(\xi)
    = \widehat{f}(\xi)
    \quad \text{for all } f \in L^1(\R).
  \]
  Take $f(x) = e^{-x^2}$ with
  $\widehat{f}(\xi) = e^{-\xi^2}$ and
  note that $\widehat{f}(\xi)$ is
  everywhere nonzero. Then
  $\widehat{\delta}(\xi) = 1$ for
  all $\xi \in \R$, which contradicts the
  Riemann-Lebesgue lemma.

  The correct way to work with the
  Dirac delta is to use the measure
  \[
    \delta(E) =
    \begin{cases}
      1, & 0 \in E, \\
      0, & 0 \notin E.
    \end{cases}
  \]
  One can then integrate against
  the measure $\delta$ to achieve a
  similar effect.
\end{remark}
