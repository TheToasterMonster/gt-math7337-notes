\chapter{Sept.~4 --- Approximate Identities}

\section{Properties of Approximate Identities}

\begin{theorem}
  If $\{k_{\lambda}\}$ is an approximate
  identity, then for all $f \in L^1(\R)$,
  \[
    \lim_{\lambda \to \infty} \|f * k_{\lambda} - f\|_1 = 0.
  \]
  That is, $f * k_{\lambda} \to f$ in $L^1$-norm.
\end{theorem}

\begin{proof}
  We have already seen that
  $f * k_{\lambda} \in L^1(\R)$. Then
  \[
    \|f - f * k_{\lambda}\|_1
    = \int |f(x) - (f * k_{\lambda})(x)| \, dx
    = \int \left|f(x) \int k_{\lambda}(t)\, dt - \int f(x - t)k_\lambda(t)\, dt\right| dx,
  \]
  where we used that
  $\int k_{\lambda}(t)\, dt = 1$.
  Collecting terms and taking absolute
  values inside,
  \[
    \|f - f * k_{\lambda}\|_1
    \le \iint |f(x) - f(x - t)| |k_{\lambda}(t)|\, dt dx.
  \]
  By Tonelli's theorem, we can
  exchange orders to get
  \[
    \|f - f * k_{\lambda}\|_1
    \le \int |k_{\lambda}(t)| \left(\int |f(x) - T_t f(x)|\, dx\right) dt
    = \int |k_{\lambda}(t)| \|f - T_t f\|_1 dt.
  \]
  We split this integral into two parts:
  \[
    \|f - f * k_{\lambda}\|_1
    \le \int_{|t| < \delta} |k_{\lambda}(t)| \|f - T_t f\|_1 dt
    + \int_{|t| \ge \delta} |k_{\lambda}(t)| \|f - T_t f\|_1 dt.
  \]
  By the strong continuity of translation,
  we know that $\lim_{t \to 0} \|f - T_t f\|_1 = 0$,
  so for any $\epsilon > 0$, there exists
  $\delta > 0$ such that
  $|t| < \delta$ implies
  $\|f - T_t f\|_1 < \epsilon$.
  This lets us estimate the first integral:
  \[
    \|f - f * k_{\lambda}\|_1
    \le \epsilon \int_{|t| < \delta} |k_{\lambda}(t)|\, dt
    + \int_{|t| \ge \delta} |k_{\lambda}(t)| \|f - T_t f\|_1 dt.
  \]
  For the second integral, we can use
  $\|f - T_t f\|_1 \le \|f\|_1 + \|T_t f\|_1 = 2\|f\|_1$ to get
  \[
    \|f - f * k_{\lambda}\|_1
    \le \epsilon \int_{|t| < \delta} |k_{\lambda}(t)|\, dt
    + 2\|f\|_1 \int_{|t| \ge \delta} |k_{\lambda}(t)|\, dt
    \le \epsilon K + 2 \|f\|_1 \epsilon
  \]
  where $K = \sup_{\lambda} \|k_{\lambda}\|_1 < \infty$ and
  $\lambda$ is large enough (as
  $\int_{|t| \ge \delta} |k_{\lambda}(t)|\, dt \to 0$).
  So $\|f - f * k_{\lambda}\|_1 \to 0$.
\end{proof}

\begin{exercise}
  Show that for $1 \le p < \infty$,
  we still have
  $\|f - f * k_{\lambda}\|_{p} \to 0$
  as $\lambda \to \infty$ for
  $f \in L^p(\R)$.
  For $p = \infty$, show that if
  $f \in C_0(\R)$, then
  $\|f - f * k_{\lambda}\|_{\infty} \to 0$
  as $\lambda \to \infty$, that is
  $f * k_{\lambda} \to f$ uniformly.
\end{exercise}

\begin{exercise}
  Show that if $f \in C_b(\R)$, then for every
  compact set $K \subseteq \R$,
  \[
    \lim_{\lambda \to \infty}
    \|(f - f * k_{\lambda}) \chi_K\|_{\infty} = 0.
  \]
\end{exercise}

\begin{definition}
  A function $f$ is \emph{H\"older continuous} with exponent $\alpha > 0$ if
  \[
    |f(x) - f(y)| \le  K |x - y|^{\alpha}
  \]
  for some constant $K$ and all $x, y$.
  If $\alpha = 1$, then we say that
  $f$ is \emph{Lipschitz}.
\end{definition}

\begin{remark}
  If $f$ is H\"older continuous with
  exponent $\alpha > 1$, then the mean value
  theorem implies that $f$ is constant.
  Thus the interesting
  range for
  H\"older continuity
  is $0 < \alpha \le 1$.
\end{remark}

\begin{exercise}
  Let $f$ be bounded and
  H\"older continuous with exponent $0 < \alpha \le 1$,
  then show that
  \[
    f * k_{\lambda} \to f \quad \text{uniformly on $\R$}.
  \]
\end{exercise}

\begin{remark}
  If $f$ is differentiable and $f'$
  is bounded, then $f$ is Lipschitz.
\end{remark}

\begin{remark}
  Recall the \emph{Lebesgue differentiation theorem}, which
  says that if $f \in L^1_{\loc}(\R)$, then
  \[
    (f * g_T)(x)
    = \frac{1}{2T} \int_{x - T}^{x + T} f(t)\, dt
    \longrightarrow f(x)
    \quad \text{for a.e. $x$}.
  \]
  where $g_T = \chi_{[-T, T]}/(2T)$. The
  points where the limit holds are called
  the \emph{Lebesgue points} of $f$.
\end{remark}

\begin{theorem}
  Assume $k$ is bounded and compactly
  supported and
  $\int k = 1$. Set
  $k_\lambda(x) = \lambda k(\lambda x)$
  for $\lambda > 0$.
  Then for any $f \in L^1(\R)$,
  \[
    f * k_\lambda \to f \quad
    \text{pointwise a.e.}
  \]
  Moreover, the pointwise limit holds
  at every Lebesgue point of $f$.
\end{theorem}

\begin{proof}
  Assume $\supp(k) \subseteq [-R, R]$.
  We can write
  \begin{align*}
    \lim_{\lambda \to \infty}
    |f(x) - (f * k_{\lambda})(x)|
    &= \lim_{\lambda \to \infty}
    \left|f(x) \int k_\lambda(x - t)\, dt - \int f(x) k_{\lambda}(x - t)\, dt\right| \\
    &\le \lim_{\lambda \to \infty}
    \int |f(x) - f(t)| \lambda |k(\lambda x - \lambda t)|\, dt \\
    &\le \lim_{\lambda \to \infty}
    \lambda \int_{x - R / \lambda}^{x + R / \lambda} |f(x) - f(t)| |k(\lambda x - \lambda t)|\, dt.
  \end{align*}
  Making a change of variables
  $T = R / \lambda$, we have
  \[
    \lim_{\lambda \to \infty}
    |f(x) - (f * k_{\lambda})(x)|
    \le \lim_{T \to 0}
     \frac{1}{2T}\int_{x - T}^{x + T} |f(x) - f(t)| \, dt
     \cdot \|k\|_{\infty}
     = 0
  \]
  for every Lebesgue point $x$
  by the Lebesgue differentiation theorem.
\end{proof}

\section{Density Results and Smooth Urysohn Lemma}
\begin{theorem}
  $C_c^m(\R)$ is dense in $L^p(\R)$
  for $m > 0$ and $1 \le p < \infty$.
\end{theorem}

\begin{proof}
  Fix $\epsilon > 0$. Choose
  $k \in C_c^m(\R)$ with
  $\int k = 1$, and
  set $k_{\lambda}(x) = \lambda k(\lambda x)$.
  Note that there exists a compactly
  supported $g \in L^p(\R)$ with
  $\|f - g\|_p < \epsilon$ (e.g.
  take $g = f \chi_{[-R, R]}$ for large
  enough $R$, this works since $f \chi_{[-R, R]}$ converges
  pointwise to $f$ as $R \to \infty$ and
  is dominated by $f$, so the dominated
  convergence theorem implies that
  $f \chi_{[-R, R]} \to f$ in $L^p$-norm).
  Then note that
  $g * k_{\lambda} \in C_c^m(\R)$ and
  $g * k_{\lambda} \to g$ in $L^p$-norm, so
  there exists $\lambda$ such that
  $\|g - g * k_{\lambda}\|_p < \epsilon$.
  Thus
  \[
    \|f - g * k_{\lambda}\|_p
    \le \|f - g\|_p + \|g - g * k_{\lambda}\|_p
    < 2\epsilon
  \]
  which implies the desired result.
\end{proof}

\begin{corollary}
  $C_c^{\infty}(\R)$ is dense in $L^p(\R)$
  for $1 \le p < \infty$.
\end{corollary}

\begin{remark}
  The above proof would work for $m = 0$
  but becomes circular: The step
  $g * k_{\lambda} \in C_c^m(\R)$
  relies on the strong continuity of
  translation, which we proved by first
  showing it for $C_c(\R)$ and then by
  an extension by density to
  $L^p(\R)$. In particular, we needed to
  already know that $C_c(\R)$ is dense
  in $L^p(\R)$.
\end{remark}

\begin{prop}[$C^\infty$ Urysohn's lemma]
  If $K \subseteq \R$ is compact and
  $U \subseteq K$ is open, then
  there exists $f \in C_c^{\infty}(\R)$
  such that $0 \le f \le 1$, $f = 1$ on
  $K$, and $f = 0$ on $U^c$.
\end{prop}

\begin{proof}
  Since $K$ is compact and $U^c$ is closed,
  we have
  \[
    d = \dist(K, U^c)
    = \inf\{|x - y| : k \in K, y \notin U\}
    > 0.
  \]
  Set $V = \{y \in \R : \dist(y, K) < d / 3\}$,
  and choose any $k \in C_c^\infty(\R)$
  such that $\supp(k) \subseteq [-d/3, d/3]$
  and $\int k = 1$.
  Take $f = k * \chi_V \in C_c^\infty(\R)$,
  which has $\supp(f) \subseteq \supp(k) + V \subseteq U$. If $x \in K$, then
  \[
    f(x) = \int_V k(x - y)\, dy
    = \int k = 1.
  \]
  One can check that $0 \le f \le 1$
  and $f = 0$ on $U^c$
  as an exercise, which would prove
  the result.
\end{proof}
