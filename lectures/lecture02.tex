\chapter{Aug.~21 --- The Riemann-Lebesgue Lemma}

\section{Properties of the Fourier Transform}

\begin{definition}
  Define the following operators:
  \begin{enumerate}
    \item \emph{Translation}:
      $T_a f(x) = f(x - a)$ for $a \in \R$;
    \item \emph{Modulation}:
      $M_b f(x) = e^{2\pi i b x} f(x)$
      for $b \in \R$;
    \item \emph{Dilation}:
      $f_{\lambda}(x) = \lambda f(\lambda x)$
      for $\lambda > 0$;
    \item \emph{Involution}:
      $\widetilde{f}(x) = \overline{f(-x)}$.
  \end{enumerate}
\end{definition}

\begin{remark}
  Translation and modulation are
  isometries on $L^p(\R)$ for any $p$.
  Dilation
  as defined above is $L^1$-normalized, so
  it is only an isometry on $L^1(\R)$.
\end{remark}

\begin{exercise}
  If $f \in L^1(\R)$, then
  \begin{enumerate}
    \item $(T_a f)^\wedge(\xi) = (M_{-a} \widehat{f})(\xi) = e^{-2\pi i \xi a} \widehat{f}(\xi)$;
    \item $(M_b f)^\wedge(\xi) = (T_b \widehat{f})(\xi) = \widehat{f}(\xi - b)$;
    \item $(f_\lambda)^\wedge(\xi) = \lambda (f_{1 / \lambda})^\wedge(\xi) = \widehat{f}(\xi / \lambda)$;\footnote{Note that the result is an $L^\infty$-normalized dilation.}
    \item $(\overline{f})^\wedge(\xi) = (\widehat{f})^{\sim}(\xi) = \overline{\widehat{f}(-\xi)}$;
    \item $(\widetilde{f})^\wedge(\xi) = \overline{\widehat{f}(\xi)}$.
  \end{enumerate}
\end{exercise}

\section{The Riemann-Lebesgue Lemma}
\begin{definition}
  Let $C_c(\R)$ be the space of continuous
  functions with compact support. For a continuous function, the \emph{support} of $f$, denoted $\supp(f) = \overline{\{x \in \R : f(x) \ne 0\}}$.
  So for a continuous function $f$,
  $\supp(f)$ is compact if and only if
  $f = 0$ outside some finite interval.
\end{definition}

\begin{theorem}
  $C_c(\R)$ is dense in $L^p(\R)$
  for $1 \le p < \infty$. In other words,
  \begin{enumerate}
    \item the closure of $C_c(\R)$ in $L^p(\R)$
      is all of $L^p(\R)$;
    \item for any $f \in L^p(\R)$
      and $\epsilon > 0$, there
      exists $g \in C_c(\R)$ such that
      $\|f - g\|_p < \epsilon$;
    \item if $f \in L^p(\R)$, then there
      exists $g_n \in C_c(\R)$ such that
      $g_n \to f$ in $L^p$-norm, i.e.
      $\|g_n - f\|_p \to 0$.
  \end{enumerate}
  For $p = \infty$, $C_c(\R)$ is
  dense in $C_0(\R)$ with respect to
  the $L^\infty$-norm (this is the same
  as the uniform norm for continuous
  functions).
\end{theorem}

\begin{proof}
  We sketch the proof. First approximate
  $f \in L^p(\R)$ by a simple
  function (one that takes only finitely
  many distinct values)
  $\phi = \sum_{k = 1}^N c_k \chi_{E_k}$,
  e.g. by rounding down to the nearest
  integer multiple of $2^{-n}$. Then
  use Urysohn's lemma to approximate
  $\chi_{E_k}$ by a continuous function.
\end{proof}

\begin{exercise}
  Fix $1 \le p < \infty$. Prove that
  if $f \in L^p(\R)$, then
  $\lim_{a \to 0} \|f - T_a f\|_p = 0$.
  We say that translation is
  \emph{strongly continuous} on $L^p(\R)$.
  For $p = \infty$, use
  $C_0(\R)$ and the uniform norm instead.
\end{exercise}

\begin{lemma}[Riemann-Lebesgue lemma]
  If $f \in L^1(\R)$, then
  $\widehat{f} \in C_0(\R)$,
\end{lemma}

\begin{proof}
  We have already seen that $\widehat{f}$
  is continuous. So it
  suffices to show decay at $\infty$.
  Write
  \[
    \widehat{f}(\xi)
    = - \int f(x) e^{-2\pi i \xi x}
    e^{-2\pi i \xi (1 / 2\xi)}\, dx
    = -\int_{-\infty}^\infty f(x) e^{-2\pi i \xi (x + 1 / 2\xi)}\, dx.
  \]
  Now make the change of variables
  $x \mapsto x - 1 / 2\xi$, so we get
  \[
    \widehat{f}(\xi)
    = - \int_{-\infty}^\infty f\left(x - \frac{1}{2\xi}\right) e^{-2\pi i \xi x}\, dx
    = - \int T_{1 / 2\xi} f(x) e^{-2\pi i \xi x}\, dx.
  \]
  Taking an average with the usual
  expression for $\widehat{f}(\xi)$, we
  have
  \[
    \widehat{f}(\xi)
    = \frac{1}{2} \int (f(x) - T_{1 / 2\xi} f(x)) e^{-2\pi i \xi x}\, dx.
  \]
  Taking absolute values, we obtain
  \[
    |\widehat{f}(\xi)|
    \le \frac{1}{2} \int |f(x) - T_{1 / 2\xi} f(x)|\, dx
    = \frac{1}{2} \|f - T_{1 / 2\xi} f\|_1
    \xrightarrow[\xi \to \pm \infty]{}
    0
  \]
  by the strong continuity of translation
  on $L^1(\R)$.
\end{proof}

\begin{exercise}
  The following is an alternative proof
  of the Riemann-Lebesgue lemma.
  Recall that we have $\widehat{\chi}_{-T, T} = d_{2\pi T} \in C_0(\R)$.
  By taking translations and dilations, we
  see that $\widehat{\chi}_{[a, b]} \in C_0(\R)$.
  Consider \emph{really simple functions}
  $\phi = \sum_{k = 1}^N c_k \chi_{[a_k, b_k]}$,
  and by linearity we can write
  \[
    \widehat{\phi}
    = \sum_{k = 1}^N c_k \widehat{\chi}_{[a_k, b_k]} \in C_0(\R).
  \]
  Note that really simple functions are also
  dense in $L^p(\R)$ for $1 \le p < \infty$.
  So if $f \in L^1(\R)$, there exist
  really simple $\phi_n \to f$ in $L^1$-norm.
  On the Fourier side, we have
  \[
    \|\widehat{f} - \widehat{\phi}_n\|_\infty
    \le \|f - \phi_n\|_1 \longrightarrow 0.
  \]
  Since $\phi_n \to \widehat{f}$ uniformly
  and $C_0(\R)$ is a Banach space,
  we conclude $\widehat{f} \in C_0(\R)$.
  Fill in the details.
\end{exercise}

\section{Position and Momentum Operators}

\begin{definition}
  The \emph{position operator} $P : L^1(\R) \to L^1(\R)$ is
  given by $P f(x) = x f(x)$.
  Note that $P$ is unbounded on $L^1(\R)$
  (in fact, $P$ is not defined on all
  of $L^1(\R)$).
  Restrict $P$ to the domain
  \[
    D_P = \{f \in L^1(\R) : xf(x) \in L^1(\R)\},
  \]
  which is dense in $L^1(\R)$.
  Note that $D_P$ cannot be bounded
  as it does not admit an extension
  to $L^1(\R)$.
\end{definition}

\begin{exercise}
  Show that
  $\sup_{\|f\|_1 = 1, f \in D_P} \|P f\|_1 = \infty$.
\end{exercise}

\begin{definition}
  The \emph{momentum operator}
  $M : L^1(\R) \to L^1(\R)$ is given by
  $M f = f' / 2\pi i$. Similarly,
  $M$ is unbounded and defined only on a
  dense subset of $L^1(\R)$.
\end{definition}

\begin{remark}
  We have the relation
  $(Mf)^\wedge(\xi) = \xi P \widehat{f}(\xi)$,
  whenever the statement makes sense.
\end{remark}

\section{An Open Problem}

\begin{conjecture}[HRT conjecture]
  Assume $g$ is not zero a.e.,
  $a_k, b_k$ are distinct, and
  consider finite linear combinations of
  translations and modulations
  of $g \in L^2(\R)$ of the following form:
  \[
    \sum_{k = 1}^N c_k e^{2\pi i b_k x} g(x - a_k). \tag{$*$}
  \]
  If $(*) = 0$, then must it be that
  $c_1 = \cdots = c_N = 0$? In other
  words, are these linearly
  independent?
\end{conjecture}

\begin{remark}
  Consider the special case $b_k = 0$
  for every $k$, so
  $\sum c_k T_{a_k} g = \sum c_k g(x - a_k) = 0$ a.e. Then
  \[
    \left(\sum c_k T_{a_k} g\right)^\wedge
    = \sum c_k M_{-a_k} \widehat{g}
    = \left(\sum_{k = 1}^N c_k e^{-2\pi i a_k \xi}\right)
    \widehat{g}(\xi) = 0.
  \]
  Since $\widehat{g}$ is not zero a.e., we
  must have
  $\sum_{k = 1}^N c_k e^{-2\pi i a_k \xi} = 0$,
  which implies $c_k = 0$ for all $k$.
  In particular, this means that
  translations alone are linearly independent
  (the same is true for modulations alone).
\end{remark}

\begin{remark}
  The general case of the HRT conjecture
  is still open. Note that after taking a
  Fourier transform, we end up with
  the same problem, just for $\widehat{g}$
  instead of $g$.
\end{remark}

\section{Convolution}

\begin{definition}
  If $f, g$ are measurable on $\R$, their
  \emph{convolution} is (formally)
  \[
    (f * g)(x)
    = \int_{-\infty}^\infty
    f(y) g(x - y)\, dy.
  \]
\end{definition}

\begin{remark}
  Note that if $g = \chi_{-T, T} / 2T$,
  then
  \[
    \int_{-\infty}^\infty
    f(y) g(x - y)\, dy
    = \frac{1}{2T} \int_{x - T}^{x + T} f(y) dy
    = \mathrm{Avg}_{[-T, T]} f,
  \]
  so we can see convolution as a
  averaging or smoothing operation
  (also known as \emph{mollification}).
\end{remark}

\begin{remark}
  We would like to show
  $f, g \in L^1(\R)$ implies
  $f * g \in L^1(\R)$. Note that
  $(f * g)^\wedge = \widehat{f} \widehat{g} \in C_0(\R)$,
  since $C_0(\R)$ is closed under
  multiplication, even though $L^1(\R)$
  is not.
\end{remark}
