\chapter{Aug.~19 --- The Fourier Transform}

\section{The Fourier Transform on \texorpdfstring{$L^1(\R)$}{L1(R)}}

All integrals will be taken over $\R$
unless otherwise specified.

\begin{definition}
  The \emph{Fourier transform} of
  $f \in L^1(\R)$ is
  \[
    \widehat{f}(\xi)
    = \int f(x) e^{-2\pi i \xi x}\, dx,
    \quad \xi \in \R.
  \]
\end{definition}

\begin{remark}
  Note that by the triangle inequality,
  \[
    |\widehat{f}(\xi)|
    \le \int |f(x) e^{-2\pi i \xi x}|\, dx
    = \int |f(x)|\, dx
    = \|f\|_{1}
    < \infty,
  \]
  so $\widehat{f}(\xi)$ exists
  for all $\xi \in \R$ (in fact,
  $\widehat{f}$ is continuous).
\end{remark}

\begin{remark}
  The Fourier transform is an
  operator $\F : L^1(\R) \to L^\infty(\R)$
  as
  $\|\widehat{f}\|_\infty = \esssup_{\xi \in \R} |\widehat{f}(\xi)| \le \|f\|_1$.
  This is linear in $f$.
  The \emph{operator norm} of
  $\F$ is
  \[
    \|\F\| = \|\F\|_{L^1 \to L^\infty}
    = \sup_{\|f\|_1 = 1} \|\widehat{f}\|_\infty
    \le \sup_{\|f\|_1 = 1} \|f\|_1
    = 1,
  \]
  so $\F$ is a bounded linear operator.
  However, $\mathcal{F}$ is
  not isometric (norm-preserving)
  in general.
\end{remark}

\begin{remark}
  Observe that
  \[
    \widehat{f}(0)
    = \int f(x) e^{-2\pi i \cdot 0 \cdot x}\, dx
    = \int f(x)\, dx.
  \]
  So if $f \ge 0$ and we normalize
  $f$ so that $\widehat{f}(0) = 1$, then
  we have
  \[
    |\widehat{f}(\xi)|
    \le \int f(x)\, dx = \widehat{f}(0),
  \]
  and so $\|\widehat{f}\|_\infty = \esssup_{\xi \in \R} |\widehat{f}(\xi)| \le 1$.
  This is one particular
  case where $\F$ does preserve the norm.
\end{remark}

\begin{definition}
  For $r \ne 0$, \emph{dilation} of
  $f$ by $r$ is $f_r(x) = rf(rx)$.
  Note that $\|f_r\|_1 = \|f\|_1$.
\end{definition}

\begin{example}
  The \emph{Dirichlet function}
  is $d(\xi) = \sin(\xi) / \pi \xi \in C_0(\R)$.\footnote{Recall that $C_0(\R)$ is the space of continuous functions $f : \R \to \C$ such that $\lim_{x \to \pm \infty} f(x) = 0$.}
  Note that $d \notin L^1(\R)$.
  We can also define the
  \emph{sinc} function as
  $\sinc \xi = \sin(\pi \xi) / (\pi \xi) = d\pi(x)$.

  However, $d$ is the Fourier transform
  of a function in $L^1(\R)$.
  Consider the
  \[
    \chi_{-[T, T]}(x) =
    \begin{cases}
      1 & \text{if } |x| \le T, \\
      0 & |x| > T.
    \end{cases}
  \]
  Note that $\chi_{-[T, T]} \in L^1(\R)$.
  Then we have
  \[
    \widehat{\chi}_{-[T, T]}(\xi)
    = \int_{-T}^{T} e^{-2\pi i \xi x}\, dx
    = \left.\frac{e^{-2\pi i \xi x}}{-2\pi i \xi}\right|_{-T}^T
    = \frac{\sin(2\pi T \xi)}{\pi \xi}
    = d_{2\pi T}(\xi),
  \]
  so we see that
  $\widehat{\chi}_{-[T, T]} \in C_0(\R) \subsetneq L^\infty(\R)$.
\end{example}

\begin{remark}
  We will see in general that
  $\mathcal{F} : L^1(\R) \to C_0(\R)$,
  this is the Riemann-Lebesgue lemma.
  The image of $\F$ is a proper
  dense subspace of $C_0(\R)$, which
  implies that $\F^{-1}$ must be unbounded
  as a linear operator by Banach space
  theory.
\end{remark}

\begin{prop}
  If $f \in L^1(\R)$, then $\widehat{f}$
  is uniformly continuous on $\R$, i.e.
  \[
    \|\widehat{f} - T_\eta\widehat{f}\|_\infty
    = \sup_{\xi \in \R} |\widehat{f}(\xi) - \widehat{f}(\xi - \eta)|
    \xrightarrow{\eta \to 0} 0,
  \]
  where $T_\eta \widehat{f}(\xi) = \widehat{f}(\xi - \eta)$.
\end{prop}

\begin{proof}
  We can write the difference as
  \[
    |\widehat{f}(\xi) - \widehat{f}(\xi - \eta)|
    = \left|\int f(x) (e^{-2\pi i \xi x} - e^{-2\pi i (\xi - \eta)x})\, dx\right|
    \le \int |f(x)| |e^{-2\pi i \xi x} - e^{-2\pi i (\xi - \eta)x}|\, dx.
  \]
  Note that $f \in L^1(\R)$ and
  $|e^{-2\pi i \xi x} - e^{-2\pi i (\xi - \eta)x}| = |1 - e^{2\pi i \eta x}| \to 0$
  as $\eta \to 0$
  independent of $\xi$, so the statement
  follows from the dominated convergence
  theorem (the integrand is dominated
  by $2f$).
\end{proof}

\section{Motivation for the Fourier Transform}

\begin{remark}
  We will define the \emph{inverse Fourier transform}
  of $f \in L^1(\R)$ as
  \[
    \widecheck{f}(x)
    = \int f(x) e^{2\pi i \xi x}\, d\xi.
  \]
  Note that $\widecheck{f}(\xi) = \widehat{f}(-\xi)$.
  With enough assumptions, this is
  an inverse to the Fourier transform.
\end{remark}

\begin{prop}[Fourier inversion formula]
  If $f, \widehat{f} \in L^1(\R)$,
  then
  \[
    f(x) = (\widehat{f})^\vee(x)
    = \int \widehat{f}(\xi) e^{2\pi i \xi x}\, d\xi.
  \]
\end{prop}

\begin{remark}
  Note that $e_\xi(x) = e^{2\pi i \xi x} = \cos 2\pi \xi x + i \sin 2\pi \xi x$
  and $e_\xi : \R \to S^1 = \{ z \in \C : |z| = 1 \}$.
  We have
  $e_\xi(x + y) = e_\xi(x) e_\xi(y)$, so
  $e_\xi$ is a homomorphism, and it is
  also continuous. Thus $e_\xi$ is
  a \emph{character} on $\R$ (in fact,
  every character on $\R$ is of the form
  $e_\xi$ for some $\xi$). One
  can use this idea to define Fourier
  transforms in much more general settings.
\end{remark}

\begin{remark}
  The Fourier transform decomposes a
  function $f$ into the pure harmonics
  $e_\xi$, and the inversion formula says
  that we can recover $f$ as a ``sum''
  of these pure harmonics.
\end{remark}
